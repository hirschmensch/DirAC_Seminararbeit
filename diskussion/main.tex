Die gute Bewertung des Referenzsignals (FOA: first order ambisonics) ist ein überraschendes Ergebnis dieses Hörversuchs. Aus der Theorie ist zu erwarten, dass durch die Upmixing-Verfahren eine Verbesserung der räumlichen Bewertungen eintritt.
Dies kann wieder mit der Art der Formulierung der Fragestellungen erklärt werden.
Die Upmixing-Verfahren schärfen den gerichteten Signalanteil und ermöglichen eine Erhöhung der Diffusität des diffusen Anteils.
Die Fragestellungen zielten jedoch auf Klangqualität und räumliche Qualität ab.
Da FOA als Referenzsignal als einziges nicht von einem Algorithmus verändert wurde, ist auch zu erwarten, dass die Bewertung nach möglichen hörbaren Artefakten (Klangqualität) am besten ausfällt.
Weiters kann argumentiert werden, dass eine künstliche Dekorrelation des Diffusanteils synthetischer wahrgenommen wird, als das originale aufgenommene Diffussignal der Aufnahme, obwohl FOA die räumliche Wiedergabequalität einschränkt.
Nach unseren Kriterien schneidet HARPEX als bester Upmixing-Algorithmus und COMPASS klar als schlechtester ab.

Die Bewertungen der verschiedenen DirAC Implementierungen liegen dazwischen und sind nicht statistisch signifikant unterschiedlich.
Tendentiell ist LSFDN (DirAC dekodiert auf die physische Lautsprecheranordnung mit FDN zur Synthese des Diffussignals) als bester DirAC-Algorithmus bewertet worden.
Ein interessantes Ergebnis ist, dass LSFDN besser bewertet wurde als LSDecorr (originale Diffussignalsynthese nach \cite{pulkki}).
Der Unterschied der Klangqualität dieser zwei Algorithmen ist statistisch signifikant (p=?????TODO).
