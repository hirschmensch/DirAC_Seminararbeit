\documentclass[12pt, a4paper]{article}
\usepackage{iem} %enthaelt viele nuetzliche usepackages und definiert 
% ein paar Reformatierungen
\hyphenation{}

\begin{document}
\selectlanguage{ngerman}

\title{Untersuchung verschiedener Upmixingmethoden für DirAC}

\author{Manuel Planton, BSc \\ Michael Hirschmugl, BSc\\\\\small{Betreuung: Dr. Franz Zotter, Dr. Matthias Frank}}

\markboth{}{M. Planton, M. Hirschmugl: DirAC Upmixing}

\doctype{Seminararbeit aus Algorithmen in Akustik und Computermusik 2}

%\date{Juni 2008}

\maketitle
\newpage
\pagestyle{empty}
\hspace{1cm}\vspace{3cm}

\hspace{1cm}\vspace{1cm}

\begin{abstract}
   Kurzfassungstext. Beispielzitate Buch~\cite{Williams}, in einem Konferenzband\cite{Avizienis06}, Diplomarbeit~\cite{Pomberger08}, Dissertation~\cite{Li05}, Fachzeitschriftenartikel~\cite{Weinreich80}.
\end{abstract}
\newpage
\pagestyle{myheadings}
\hspace{1cm}\vspace{2cm}

\tableofcontents
\newpage

\section{Einleitung}

\section{DirAC}
    \subsection{Implementierung}
    In diesem Kapitel wird beschrieben, wie der DirAC-Algorithmus in diesem Projekt implementiert wurde. Die Implementierung liegt als Octave-Skript vor und nutzt das \textit{signal}-Package.

Zu Beginn des Skriptes wird das Eingangssignal im B-Format erster Ordnung eingelesene. Dies muss als \textit{.wav}-Datei vorliegen und genau vier Kanäle (in der Reihenfolge $w$, $x$, $y$, $z$) umfassen. Die Samplerate wird direkt aus dieser Datei ermittelt und als Variable \textit{fs} im Skript gespeichert. Anschließend wird eine VBAP-Matrix, je nach vorgegebener Lautsprecherandordnung erzeugt. Diese Matrix wird in späterer Folge verwendet, um das Eingangssignal im B-Format, auf eine bestimmte Lautsprecheranordnung zu dekodieren. Es handelt sich dabei also um eine nicht-parametrische Dekodierung, die mit einer Anordnung von virtuellen Mikrofonen zu vergleichen ist. Die Lautsprecherpositionen können dabei als sphärische Koordinaten in einer Matrix vorliegen, oder auch aus einer Textdatei mit kartesischen Koordinaten eingelesen werden. Diese Textdatei muss eine Matrix mit genau drei Spalten (Raumdimensionen) enthalten, wobei diese Spalten untereinander angeordnet sein müssen. Es befinden sich demnach genau die dreifache Anzahl der Lautsprecher als Zeilen in dieser Datei. Die Koordinaten sind dabei Meterangaben als Dezimalwerte. Das Einlesen von kartisischen Koordinaten ist speziell für die Dekodierung auf größere T-Designs äußerst nützlich.

Die vier Kanäle $w[n]$, $x[n]$, $y[n]$ und $z[n]$ des B-Format Eingangssginals werden anschließend in einer Schleife in zeitlichen Blöcken (Fenster) verarbeitet. Ein Fenster besteht dabei aus jeweils 512 Samples und wird in einer 1024-Punkt Fouriertransformation mit der Hilfe der Funktion \textit{fft()} in den Frequenzbereich transformiert. Die FFT-Länge wurde als doppelte Blocklänge gewählt um Aliasing zu vermeiden. Die zeitlichen Fenster werden noch mit einer Hanning-Fensterfunktion für die Resynthese beaufschlagt. Ein Hanning-Fenster kann in Octave mit der Funktion \textit{hanning()} erzeugt werden und einfach als Vektor mit den Zeitsignalen multipliziert werden.
        \paragraph{Analyse von Richtungs- und Diffusanteil}
        Zur Bestimmung von Richtungs- und Diffusanteil werden zunächst Schallschnelle und Energie berechnet. Die Schnallschnelle wird als Vektor $\textbf{V}_{m}[k] = [X_{m}[k], Y_{m}[k], Z_{m}[k]]$ aus den gerichteten Anteilen des B-Format Signals bestimmt. Der Schalldruck ist der omnidirektionale Anteil $W_{m}[k]$. Die Indizes $m$ und $k$ werden hier zur Kennzeichnung des Zeitfensters als Funktion der Frequenzzahl $k$ verwendet.

Der Pseudo-Schallintensitätsvektor $\textbf{I}_{m}[k]$ wird aus dem Schnellevektor und dem konjugiert komplexen Schalldruck in Gleichung \ref{eq:inten} abgeleitet, wobei hier rein der Realteil heranzogen wird, da sonst die Blindanteile des Schnellevektors das Ergebnis verfälschen \cite{pulkki}. Der Pseudo-Intensitätsvektor stellt bereits die Schalleinfallsrichtung für alle Frequenzbins einzeln dar, jedoch entgegengesetzt der Einfallsrichtung $\textbf{D}_{m}[k]$

\begin{equation}
    -\textbf{D}_{m}[k] = \textbf{I}_{m}[k] = \Re(W_{m}[k]^{*} \cdot \textbf{V}_{m}[k]) .
    \label{eq:inten}
\end{equation}

Die Schallenergie $E_{m}[k]$ wird mit Hilfe der Gleichung

\begin{equation}
    E_{m}[k] = \frac{|W_{m}[k]|^2+||\textbf{V}_{m}[k]||^2}{2}
    \label{eq:energy}
\end{equation}

ausgewertet.

Um Sprünge in der Lautsprecherzuordnung von gerichteten Signalen bei raschen Bewegungen zu vermeiden, wird der Pseudo-Schallintensitätsvektor zusätzlich durch eine zeitliche Mittelwertbildung geglättet. Diese Glättung kann im Skript mit einer frequenzabhängigen Zeitkonstante vorgegeben werden. Darüberhinaus wird auch der Energievektor zeitlich geglättet. Anschließend wird der Pseudo-Schallintensitätsvektor verwendet, um die Einfallsrichtungen in sphärischen Koordinaten zu bestimmen.

Die Diffusität $\psi_{m}[k]$ kann schlussendlich aus dem Verhältnis des Betrags des Pseudo-Schallintensitätsvektors zur Energie ermittelt werden

\begin{equation}
    \psi_{m}[k] = \sqrt{1 - \frac{||\mathbb{E}(\textbf{I}_{m}[k])||}{\mathbb{E}(E_{m}[k])}} .
    \label{eq:diff}
\end{equation}

 %Der Erwartungswert entspricht hier den (zeitlich) gemittelten Vektoren.

%\begin{figure}[!ht]
%  \centering
%  \includegraphics[width=0.7\textwidth]{implementierung/plots/flow.png}
%  \label{fig:flow}
%  \caption{Flussdiagram des DirAC Algorithmus in Octave \cite{seminar2016}}
%\end{figure}

        \paragraph{Upmixing auf Lautsprecherandordnung}
        

%Zur Dekodierung auf eine bestimmte Lautsprecheranordnung wird eine Matrix aus virtuellen Mikrofonen berechnet, woraus sich eine Tabelle mit Gain-Werten ergibt. Diese Tabelle besteht aus einer Spalte für jeden Lautsprecherkanal und kann die einzelnen Frequenzbins in einem VBAP-Verfahren den enstprechenden Lautsprechern zuordnen. In dieser Seminararbeit wurden dabei zwei Ansätze für die Dekodierung verglichen: Einerseits die direkte Dekodierung auf die physische Lautsprecheranordnung des Wiedergabesystems, und andererseits ein allgemeiner Ansatz womit die Dekodierung auf eine t-Design Lautsprecheranordnung erfolgt. Dieser Dekodierungsansatz hat den Vorteil, dass die Wiedergabeanordnung zum Zeitpunkt des Upmixings nicht vorgegeben sein muss. Auch für die Klangqualität ergeben sich daraus Vorteile, welche im Hörversuch besprochen werden.

Im Frequenzbereich wird jeder spektrale Block mit dem Sampling-Dekoder dekodiert, um die vier Eingangskanäle (B-Format) in eine höhere Kanalzahl zu wandeln.
In dieser Seminararbeit wurden dabei zwei Ansätze für die Dekodierung verglichen: Einerseits die direkte Dekodierung auf die physische Lautsprecheranordnung des Wiedergabesystems, und andererseits eine allgemeine Dekodierung auf ein t-Design. Dieser Dekodierungsansatz hat den Vorteil, dass die physische Lautsprecheranordnung zum Zeitpunkt des Upmixings nicht gegeben sein muss.

Bei einem 9-Design (Abb. \ref{fig:tdesign_image}) ergibt sich eine sphärische Anordnung von 48 virtuellen Lautsprechern, welche eine optimale Kodierung/Dekodierung des B-Formats vierter Ordnung erlaubt.

\begin{figure}[!ht]
  \centering
  \includegraphics[width=0.35\textwidth]{implementierung/plots/t-design.png}
  \caption{9-Design T-Design \cite{ambi-book}}
  \label{fig:tdesign_image}
\end{figure}

%Zur Dekodierung wird in Octave also eine Matrix mit Lautsprechergewichten für Schalleinfallsrichtungen erstellt. Somit kann einer gegebenen frequenzabhängigen Richtung (bestehend aus Azimuth und Elevation), ein bestimmter Gain-Wert am entsprechenden Lautsprecherkanal zugeordnet werden.

        \paragraph{Trennung von Diffus- und Richtungsanteil}
        Die Trennung von Diffus- und Richtungsanteil erfolgt im Frequenzbereich, wobei beide Anteile aus dem omindirektionalen Anteil des B-Format Signals durch gewichtete Filterung erzeugt werden. Prinzipiell wird für den gerichteten Anteil ein spektraler Block im Frequenzbereich mit zwei Matrizen multipliziert: Einerseits mit der VBAP-Tabelle um die Richtung einem Lautsprecher zuzuordnen, und andereseits ebenfalls mit dem frequenzabhängigen Diffusitätsvektor. Somit werden Frequenzbins mit frequenzabhängigen Diffusitätswerten gewichtetet und diesen anschließend entsprechende Lautsprecheramplituden zugeordnet.

Nach der Berechnung des gerichteten Signalanteils kann auch der Diffusanteil aus dem omnidirektionalen Signal erzeugt werden. Dieser entspricht lediglich dem nicht-gerichteten Anteil des omnidirektionalen Signals und wird daher aus der Filterkurve des Direktanteils bestimmt. Somit ergeben sich an dieser Stelle zwei frequenzabhängige Signalmatrizen für Diffus- und Richtungsanteil. Beide Matrizen besitzen eine Spalte für jeden Lautsprecherkanal und die Zeilenzahl entspricht der FFT-Länge.
        \paragraph{Resynthese}
        Die Resynthese entspricht einem gewöhnlichem Overlap-Add Verfahren und erzeugt mittels inverser Fast Fourier Transformation (\textit{ifft()}) wieder Zeitsignale aus den spektralen Richtungs- und Diffusanteilen. Im Zeitbereich kann an dieser Stelle weiters auch (optional) die Dekorrelation des Diffusanteils durchgeführt werden. In unserem Skript wurde zur Dekorrelation in GNU Octave ein Random Phase Algorithmus verwendet, welcher im Kapitel ~\ref{randphas} näher erläutert wird.

\section{Hörversuch}

\section{Schlussfolgerung und Ausblick}

\bibliographystyle{IEEEtranSA}
\bibliography{bib_database}

\end{document}



