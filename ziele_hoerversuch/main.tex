Prinzipiell sollten mit dem Hörversuch folgende Forschungsfragen beantwortet werden:

\begin{itemize}
	\item Bietet das Upmixing durch DirAC auf ein ambisonisches Format höherer Ordnung einen Vorteil bezüglich der Darstellung von Räumlichkeit in der Aufnahme?
	\item Ist es möglich Upmixing mit DirAC ohne negative Beeinflussung der Klangqualität durchzuführen?
	\item Welche Kombination aus Dekorrelation und Dekodierung bietet dabei die (subjektiv und objektiv) beste Wiedergabe der Klangqualität?
	\item Bietet das Upmixing auf ein allgemeines B-Format vierter Ordnung Vorteile bezüglich Klangqualität und Räumlichkeit?
	\item Wie wird ein kommerzielles Plugin wie \textit{Harpex} im Vergleich zu einer Lösung mit DirAC bewertet?
\end{itemize}

Weiters soll die Beschaffenheit des Ausgangssignals dabei einbezogen werden. Speziell geht es dabei um die Frage ob eine sehr diffuse Aufnahme ähnlich bewertet wird wie ein eher gerichtetes Signal im B-Format erster Ordnung.

In ersten Versuchen konnten wir die Algorithmen bereits selbst vergleichen und vermuteten daher, dass die Lokalisation und Darstellung der Räumlichkeit durch DirAC positiv beeinflusst werden kann, wenn dies auch durchaus von der eingesetzten Dekorrelationsmethode stark abhängig ist. Weiters wurde angenommen, dass eine bestimmte Kombination von Algorithmen durchaus vergleichbare Ergebnisse mit dem kommerziellen Harpex-Plugin bieten könnte. Speziell die Dekodierung auf das t-Design (9-Design) mit 48 virtuellen Schallquellen könnte Vorteile in der Wiedergabe bieten, da die geringen Abstände bei einer derartig dichten Anordnung virtueller Quellen zu weniger Richtungssprüngen führen.
