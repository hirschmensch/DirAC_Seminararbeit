Prinzipiell sollten mit dem Hörversuch folgende Forschungsfragen beantwortet werden:

\begin{itemize}
	\item Bietet das Upmixing durch DirAC auf ein Amisonics-Format höhrer Ordnung einen Vorteil bezüglich der darzustellenden Räumlichkeit der Aufnahme?
	\item Ist es möglich Upmixing mit DirAC ohne negative Beeinflussung der Klangqualität durchzuführen?
	\item Welche Kombination aus Dekorrelation und Dekodierung bietet dabei die (subjektiv und objektiv) beste Wiedergabe der Klangqualität?
	\item Bietet das Upmixing auf ein allgemeines B-Format vierter Ordnung Vorteile bezüglich Klangqualität und Räumlichkeit?
	\item Wie wird ein kommerzielles Plugin wie "Harpex" im Vergleich zu einer Lösung mit DirAC bewertet?
\end{itemize}

Weiters soll die Beschaffenheit des Ausgangssignals dabei einbezogen werden. Speziell geht es dabei um die Frage ob eine sehr diffuse Aufnahme ähnlich bewertet wird wie ein eher gerichtetes Signal im B-Format erster Ordnung.

In ersten Versuchen konnten wir die Algorithmen bereits selbst vergleichen und vermuten daher, dass die Lokalisation und Darstellung der Räumlichkeit durch DirAC jedenfalls positiv beeinflusst werden kann, wennauch dies durchaus von der eingesetzten Dekorrelationsmethode stark abhängig ist. Weiters wird angenommen, dass eine bestimmte Kombination von Algorithmen durchaus vergleichbare Ergebnisse mit dem kommerziellen karpex-Plugin bieten könnte. Speziell die Dekodierung auf das T-Design (9-Design) mit 48 virtuellen Schallquellen könnte Vorteile in der Wiedergaben bieten, da die geringen Abstände bei einer derartig dichten Anordnung virtueller Quellen zu einer feinen Granularität führen.