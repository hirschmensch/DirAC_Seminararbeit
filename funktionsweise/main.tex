Die Erhöhung der Lokalisationsschärfe und Verbesserung der Qualität der Räumlichkeit gschieht als erstes mittels einem Analyseschritt, in dem die Parameter \textit{Richtung} und \textit{Diffusität} des aufgenommenen Schallfeldes ermittelt werden. Anschließend kommt es zur Trennung des gerichteten Anteils vom diffusen Anteil und schließlich zur (Re-)Synthese dieser beiden Anteile, jedoch in erhöhter, räumlicher Auflösung (und Diffusität). Die Ausgangssignale werden durch Superposition der beiden Signalanteile gebildet.