DirAC (Directional Audio Coding) ist ein Algorithmus von Ville Pulkki \cite{Pulkki} zum Upmixing für Koinzidenzmikrofon-Anordnungen wie FOA (First Order Ambisonics). Das Ziel des Algorithmus ist eine Erhöhung der Lokalisationsschärfe bei gleichzeitiger Verbesserung der Qualität der räumlichen Wiedergabe. Dies gschieht mittels einem Analyseschritt, in dem die Parameter \textit{Richtung} und \textit{Diffusität} des aufgenommenen Schallfeldes ermittelt werden. Anschließend kommt es zur Trennung des gerichteten Anteils vom diffusen Anteil und schließlich zur Synthese dieser Anteile in erhöhter räumlicher Auflösung und Diffusität. Die Ausgangssignale werden durch Superposition der beiden Signalanteile gebildet.
