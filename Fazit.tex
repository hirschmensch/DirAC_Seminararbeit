Die Funktion von DirAC als Upmixingmethode für FOA wurde theoretisch erklärt und in einer Implementierung praktisch angewandt. Es wurden verschiedene Varianten des Algorithmus entwickelt, die sich in den Methoden zur Synthese des Diffusanteils und in der Ambisonics-Dekodierung unterschieden. Diese Varianten wurden mit einer Referenz (FOA) und zwei weiteren Upmixingverfahren (HARPEX, COMPASS) in einem Hörversuch verglichen.

Die Bewertungskriterien Klangqualität und räumliche Qualität im Hörversuch wiesen wahrscheinlich auf Grund der Fragestellung bzw. der Benennung der Kriterien einen hohen Zusammenhang in den Bewertungen auf. Die Referenz (FOA) wurde ähnlich gut bewertet wie der am besten bewertete Upmixingalgorithmus HARPEX. Die Ergebnisse der Klangqualität entsprechen der Erwartung, jedoch sind die Ergebnisse der räumlichen Qualität unerwartet. Die Art der Fragestellung (natürlich, synthetisch) im Hörversuch kann auch in diesem Fall eine Erklärung für dieses Resultat sein.

Somit kann mit diesem Hörversuch gezeigt werden, dass nach unseren Kriterien FOA gleichwertig mit HARPEX am besten bewertet wurden und COMPASS am schlechtesten bewertet wurde. Die verschiedenen DirAC Implementierungen liegen dazwischen.

Das Ergebnis des Hörversuchs gibt einen Hinweis darauf, dass eine DirAC Implementierung mit direkter Ambisonics-Dekodierung auf die physische Lautsprecheranordnung und einem FDN zur Diffussignal-Synthese (LSFDN) eine bessere Klangqualität erreichen kann, als die anderen DirAC Varianten in unserem Versuch.
