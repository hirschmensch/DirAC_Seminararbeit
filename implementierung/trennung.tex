%Die Trennung von Diffus- und Richtungsanteil erfolgt im Frequenzbereich, wobei beide Anteile aus dem omindirektionalen Anteil des B-Format Signals durch gewichtete Filterung erzeugt werden. Prinzipiell wird für den gerichteten Anteil ein spektraler Block im Frequenzbereich mit zwei Matrizen multipliziert: Einerseits mit der VBAP-Tabelle um die Richtung einem Lautsprecher zuzuordnen, und andereseits mit dem ebenfalls frequenzabhängigen Diffusitätsvektor. Somit werden Frequenzbins mit frequenzabhängigen Diffusitätswerten gewichtetet, und diesen anschließend entsprechende Lautsprechergewichte zugeordnet.

%Nach der Berechnung des gerichteten Signalanteils kann auch der Diffusanteil aus dem omnidirektionalen Signal erzeugt werden. Dieser entspricht lediglich dem nicht-gerichteten Anteil des omnidirektionalen Signals und wird daher aus der Filterkurve des Direktanteils bestimmt. Somit ergeben sich an dieser Stelle zwei frequenzabhängige Signalmatrizen für Diffus- und Richtungsanteil. Beide Matrizen besitzen eine Spalte für jeden Lautsprecherkanal und die Zeilenzahl entspricht der FFT-Länge.

Wie in Abb. 2 zu erkennen ist, wird nach der Fouriertransformation das B-Format 
Die Trennung von Diffus- und Richtungsanteil erfolgt im Frequenzbereich 