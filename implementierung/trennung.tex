Die Trennung von Diffus- und Richtungsanteil erfolgt aus dem omnidirektionalen Anteil des B-Format Signals im Frequenzbereich. Prinzipiell wird für den gerichteten Anteil ein Block im Frequenzbereich mit zwei Matrizen multipliziert. Einerseits mit der VBAP-Tabelle um die Richtung einem Lautsprecher zuzuordnen, und andereseits ebenfalls mit dem frequenzabhängigen Diffusheitsvektor. Somit werden Frequenzbins mit frequenzabhängigen Diffusitätswerten gewichtetet und  Lautsprecheramplituden zugeordnet.

Der Diffusanteil entspricht lediglich dem omnidirektionalen Anteil der nicht gerichtet wurde. Somit ergeben sich an dieser Stelle zwei frequenzabhängige Signalmatrizen für Diffus- und Richtungsanteil als Arrays für alle Lautsprecherkanäle.