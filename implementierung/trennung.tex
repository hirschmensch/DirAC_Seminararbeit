%Die Trennung von Diffus- und Richtungsanteil erfolgt im Frequenzbereich, wobei beide Anteile aus dem omindirektionalen Anteil des B-Format Signals durch gewichtete Filterung erzeugt werden. Prinzipiell wird für den gerichteten Anteil ein spektraler Block im Frequenzbereich mit zwei Matrizen multipliziert: Einerseits mit der VBAP-Tabelle um die Richtung einem Lautsprecher zuzuordnen, und andereseits mit dem ebenfalls frequenzabhängigen Diffusitätsvektor. Somit werden Frequenzbins mit frequenzabhängigen Diffusitätswerten gewichtetet, und diesen anschließend entsprechende Lautsprechergewichte zugeordnet.

%Nach der Berechnung des gerichteten Signalanteils kann auch der Diffusanteil aus dem omnidirektionalen Signal erzeugt werden. Dieser entspricht lediglich dem nicht-gerichteten Anteil des omnidirektionalen Signals und wird daher aus der Filterkurve des Direktanteils bestimmt. Somit ergeben sich an dieser Stelle zwei frequenzabhängige Signalmatrizen für Diffus- und Richtungsanteil. Beide Matrizen besitzen eine Spalte für jeden Lautsprecherkanal und die Zeilenzahl entspricht der FFT-Länge.

Die Trennung von Diffus- und Richtungsanteil erfolgt ebenfalls im Frequenzbereich. Der frequenzabhängige Diffusitätsvektor wird hier verwendet, um den gerichteten Anteil aus den dekodierten Signalen (mit 12 oder 48 Kanälen) zu filtern. Prinzipiell wird für den gerichteten Anteil ein spektraler Block mit zwei Matrizen multipliziert: Erstens mit dem frequenzabhängigen Diffusitätsvektor, und weiters mit der VBAP-Tabelle um die gerichteten Signale den entsprechenden Lautsprechern zuzuordnen. Somit werden Frequenzbins mit frequenzabhängigen Diffusitätswerten gewichtetet, und diesen anschließend entsprechende Lautsprechergewichte verliehen.

Nach der Berechnung des gerichteten Signalanteils kann auch der Diffusanteil aus den dekodierten Signalen erzeugt werden. Dieser entspricht dem nicht-gerichteten Anteil und wird daher aus der Filterkurve des Direktanteils bestimmt. Somit ergeben sich an dieser Stelle zwei frequenzabhängige Signalmatrizen für Diffus- und Richtungsanteil für jedes zeitliche Verarbeitungsfenster. Beide Matrizen besitzen eine Spalte für jeden Wiedergabekanal und die Zeilenzahl entspricht der FFT-Länge.