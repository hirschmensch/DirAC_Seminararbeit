Zur Bestimmung von Richtungs- und Diffusanteil werden zunächst Schallschnelle und Energie berechnet. Die Schnallschnelle wird als Vektor $\textbf{V}_{m}[k] = [X_{m}[k], Y_{m}[k], Z_{m}[k]]$ aus den gerichteten Anteilen des B-Format Signals bestimmt. Der Schalldruck ist der omnidirektionale Anteil $W_{m}[k]$. Die Indizes $m$ und $k$ werden hier zur Kennzeichnung des Zeitfensters als Funktion der Frequenzzahl $k$ verwendet.

Der Pseudo-Schallintensitätsvektor $\textbf{I}_{m}[k]$ wird aus dem Schnellevektor und dem konjugiert komplexen Schalldruck in Gleichung \ref{eq:inten} abgeleitet, wobei hier rein der Realteil heranzogen wird, da sonst die Blindanteile des Schnellevektors das Ergebnis verfälschen \cite{pulkki}. Der Pseudo-Intensitätsvektor stellt bereits die Schalleinfallsrichtung für alle Frequenzbins einzeln dar, jedoch entgegengesetzt der Einfallsrichtung $\textbf{D}_{m}[k]$

\begin{equation}
    -\textbf{D}_{m}[k] = \textbf{I}_{m}[k] = \Re(W_{m}[k]^{*} \cdot \textbf{V}_{m}[k]) .
    \label{eq:inten}
\end{equation}

Die Schallenergie $E_{m}[k]$ wird mit Hilfe der Gleichung

\begin{equation}
    E_{m}[k] = \frac{|W_{m}[k]|^2+||\textbf{V}_{m}[k]||^2}{2}
    \label{eq:energy}
\end{equation}

ausgewertet.

Um Sprünge in der Lautsprecherzuordnung von gerichteten Signalen bei raschen Bewegungen zu vermeiden, wird der Pseudo-Schallintensitätsvektor zusätzlich durch eine zeitliche Mittelwertbildung geglättet. Diese Glättung kann im Skript mit einer frequenzabhängigen Zeitkonstante vorgegeben werden. Darüberhinaus wird auch der Energievektor zeitlich geglättet. Anschließend wird der Pseudo-Schallintensitätsvektor verwendet, um die Einfallsrichtungen in sphärischen Koordinaten zu bestimmen.

Die Diffusität $\psi_{m}[k]$ kann schlussendlich aus dem Verhältnis des Betrags des Pseudo-Schallintensitätsvektors zur Energie ermittelt werden

\begin{equation}
    \psi_{m}[k] = \sqrt{1 - \frac{||\mathbb{E}(\textbf{I}_{m}[k])||}{\mathbb{E}(E_{m}[k])}} .
    \label{eq:diff}
\end{equation}

 %Der Erwartungswert entspricht hier den (zeitlich) gemittelten Vektoren.

%\begin{figure}[!ht]
%  \centering
%  \includegraphics[width=0.7\textwidth]{implementierung/plots/flow.png}
%  \label{fig:flow}
%  \caption{Flussdiagram des DirAC Algorithmus in Octave \cite{seminar2016}}
%\end{figure}
