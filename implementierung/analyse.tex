Zur Bestimmung von Richtungs- und Diffusanteil werden zunächst Schallschnelle und Energie berechnet. Die Schnallschnelle wird als Vektor $\textbf{V}_{m}[k] = [X_{m}[k], Y_{m}[k], Z_{m}[k]]$ aus den gerichteten Anteilen (der Druckgradienten-Mikrofone) des B-Format Signals bestimmt. Der Schalldruck ist schlicht der omnidirektionale Anteil $W_{m}[k]$. Die Indizes $m$ und $k$ werden hier zur Kennzeichnung des Zeitfensters $n$ als Funktion der Frequenzzahl $k$ verwendet.

Die Schallintensität $I_{m}[k]$ wird aus dem Schnellevektor und dem konjugiert komplexen Schalldruck in Gleichung \ref{eq:inten} abgeleitet, wobei hier rein der Realteil heranzogen wird, da sonst auch die Blindeinteile des Schnellevektors miteinbezogen würden. Der Intensitätsvektor stellt bereits die Schalleinfallsrichtung für alle Frequenzbins einzeln dar, jedoch entgegengesetzt der Einfallsrichtung $\textbf{D}_{m}[k]$ :

\begin{equation}
    -\textbf{D}_{m}[k] = \textbf{I}_{m}[k] = \Re(W_{m}[k]^{*} \cdot \textbf{V}_{m}[k])
    \label{eq:inten}
\end{equation}

Die Schallenergie $E_{m}[k]$ wird mithilfe von Gleichung \ref{eq:energy} ausgewertet:

\begin{equation}
    E_{m}[k] = \frac{|W_{m}[k]|^2+||\textbf{V}_{m}[k]||^2}{2}
    \label{eq:energy}
\end{equation}

Um Sprünge in der Lautsprecherzuordnung von gerichteten Signalen bei raschen Bewegungen zu vermeiden, wird der Intensitätsvektor zusätzlich durch einen Mittelwertbildung geglättet. Diese Glättung kann im Skript mit frequenzabhängiger Zeitkonstante vorgegeben werde. Zusätzlich wird auch der Energievektor geglättet. Anschließend wird der Intensitätsvektor verwendet um die Einfallsrichtung in sphärischen Koordinaten zu bestimmen.

Die \"Diffusheit\" $\psi_{m}[k]$ kann schlussendlich aus dem Vergleich des Betrags des Intensitätsvektors mit der Energie ermittelt werden (Glg. \ref{eq:diff}). Der Erwartungswert entspricht hier den (zeitlich) gemittelten Vektoren.

\begin{equation}
    \psi_{m}[k] = \sqrt{1 - \frac{||\mathbb{E}(\textbf{I}_{m}[k])||}{\mathbb{E}(E_{m}[k])}}
    \label{eq:diff}
\end{equation}

\begin{figure}[!ht]
  \centering
  \includegraphics[width=0.7\textwidth]{implementierung/plots/flow.png}
  \label{fig:flow}
  \caption{Flussdiagram des DirAC Algorithmus in Octave}
\end{figure}
