Zur Dekodierung auf eine bestimmte Lautsprecheranordnung wird eine Matrix aus virtuellen Mikrofonen berechnet, woraus sich eine Tabelle mit Gain-Werten ergibt. Diese Tabelle kann die einzelnen Frequenzbins in einem VBAP-Verfahren den enstprechenden Lautsprechern zuordnen. In dieser Seminararbeit wurden dabei zwei Ansätze für die Dekodierung verglichen: Einerseits die direkte Dekodierung auf die physische Lautsprecheranordnung des Wiedergabesystems, und andererseits ein allgemeiner Ansatz womit die Dekodierung auf eine T-Design Lautsprecheranordnung erfolgt. Dieser Dekodierungsansatz hat den Vorteil, dass die Wiedergabeanordnung zum Zeitpunkt des Upmixings nicht vorgegeben sein muss. Auch für die Klangqualität ergeben sich daraus Vorteile, welche im Hörversuch besprochen werden.

Für das B-Format vierter Ordnung wird ein sogenanntes t-Design als virtuelle Anordnung von Lautsprechern verwendet. Wird ein t-Design neunter Ordnung eingesetzt ("9-Design", Abb. \ref{fig:tdesign}) ergibt sich daraus eine sphärische Anordnung von 48 Lautsprechern, welche die verlustlose Konvertierung zwischen B-Format und einer solchen Lautsprecheranordnung bietet. Ein 9-Design ist eine Anordnung, mit der Ambisonics bis zur vierten Ordnung dekodiert werden kann.

\begin{figure}[!ht]
  \centering
  \includegraphics[width=0.35\textwidth]{implementierung/plots/t-design.png}
  \label{fig:tdesign}
  \caption{9-Design T-Design\protect\footnotemark}
\end{figure}

\footnotetext{Quelle: \cite{ambi-book}}

Zur Dekodierung wird in Octave also eine Matrix mit Lautsprechergewichten für Schalleinfallsrichtungen erstellt. Somit kann einer gegebenen Richtung (bestehend aus Azimuth und Elevation) eine Amplitude am entsprechenden Lautsprecherkanal zugeordnet werden.
