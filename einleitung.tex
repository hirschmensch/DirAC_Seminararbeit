Räumliche Klangfeldaufzeichnungen in Ambisonics 1. Ordnung sind grundsätzlich vergleichsweise einfach und günstig zu realisieren. Mit der ambisonischen Ordnung steigen jedoch auch unterschiedliche Qualitätsmerkmale der einhüllenden Wiedergabe. Dazu zählen die Sweet-Area (der günstige Abhörbereich), die Klangqualität, sowie die räumliche Auflösung der Schallrichtungen \cite{ambi-book}. Es existieren bereits einige Verfahren in Theorie und Praxis, um Ambisonics 1. Ordnung in eine höhere Ordnung zu rechnen. Prinzipiell werden solche Verfahren unter dem englischen Begriff ``Upmixing'' zusammengefasst.

V. Pulkki hat mit DirAC (Directional Audio Coding) \cite{pulkki} einen Algorithmus vorgeschlagen, der zuallererst die Trennung von gerichteten und diffusen Schallanteilen in B-Format Signalen zum Ziel hat. Weiters können gerichtete Schallereignisse anschließend auf eine beliebige Lautsprecheranordnung dekodiert werden \cite{spatial-book}, und somit ist auch Upmixing möglich.

Das kommerzielle Plugin HARPEX von S. Berge und N. Barrett stellt eine weitere Möglichkeit dar. Die Funktionsweise beruht hier auf der Kombination von parametrischen und linearen Dekodierungsverfahren \cite{harpy2} und basiert stark auf einem Transkodierungsverfahren von 1. Ordnung Ambisonics zu binauralen Signalen \cite{harpy}. Ein ähnlichen Ansatz verfolgen die COMPASS-Plugins von L. McCormack und A. Politis \cite{compy}, welche wiederum frei verfügbar sind.

Auch wenn theoretisch unterschiedlichste Methoden zur Verfügung stehen, unterscheidet sich speziell DirAC sehr stark in den Realisierungsvarianten, welche in der Synthese des dekorrelierten Diffusfeldes ausschlaggebende Qualitätsunterschiede an den Tag legen. Auch ist bisher unklar, ob die Dekodierung in ambisonisches B-Format höherer Ordnung grundsätzlich differenziert zur direkten Dekodierung auf die wiedergebende Lautsprecheranordnung bewertet wird.

In dieser Seminararbeit werden verschiedene Implementierungen des DirAC-Algorithmus und die erwähnten Plugins untereinander verglichen. Hierzu wird im Abschnitt 2 die Theorie zu DirAC erläutert. Ausgehend von einer psychoakustischen Annahme wird die allgemeine Funktionsweise des Algorithmus, sowie die spezielle Funktionsweise der DirAC-Implementierungen für diesen Versuch erklärt. Der Abschnitt schließt mit einem Überblick der verwendeten Dekorrelationsverfahren.

Die Evaluierung erfolgt in einem Hörversuch. Abschnitt 3 beschäftigt sich mit den Zielen, dem Aufbau und mit dem Konzept dieses Hörversuchs, dessen Ergebnisse schließlich präsentiert und diskutiert werden.