In dieser Seminararbeit wird die DirAC Upmixingmethode für Ambisonics Aufnahmen erster Ordnung behandelt.
Es werden verschiedene Implementierungen des Algorithmus und auch andere Methoden verglichen.
Die Bewertung der einzelnen Algorithmen erfolgte in einem Hörversuch.

Für den Vergleich verschiedener Upmixingmethoden mit dem DirAC Algorithmus wird im Kapitel 2 die Theorie zu DirAC erläutert.
Ausgehend von einer psychoakustischen Annahme wird die allgemeine Funktionsweise des Algorithmus erklärt.
Es folgt die spezielle Funktionsweise der DirAC Implementierungen für diesen Versuch.
Da wir DirAC Implementierugen mit verschiedenen Dekorrelationsverfahren und Ambisonics Dekoder-Schritten untersucht haben, schließt Kapitel 2 mit einem Überblick über die verwendeten Dekorrelationsverfahren.

Die implementierten DirAC Algorithmen wurden in einem Hörversuch evaluiert.
Kapitel 3 beschäftigt sich mit den Zielen, dem Aufbau und mit dem Konzept dieses Hörversuchs, dessen Ergebnisse schließlich präsentiert und diskutiert werden.
