Die Funktionsweise des DirAC Algorithmus basiert nach Pulkki \cite{pulkki} auf 4 psychoakustischen Annahmen, aus denen sich eine resultierenden Annahme formulieren lässt.

\begin{enumerate}
    \item Die wahrgenommene Schalleinfallsrichtung (DOA: direction of arrival) wird bestimmt von der interauralen Zeitdifferenz (ITD: interaural time difference), von der interauralen Pegeldifferenz (ILD: interaural level difference) und von monauralen cues.
    \item Die wahrgenommene \textit{Diffusität} des Schalls wird von der interauralen Kohärenz bestimmt.
    \item Die wahrgenommene Klangfarbe des Schalls hängt vom Spektrum, der ITD, der ILD und von der interauralen Kohärenz ab.
    \item Die wahrgenomme Richtung wird bestimmt von der DOA, der Diffusität und vom Spektrum des Schalls, gemessen in einer Richtung mit Zeit-/Frequenzauflösung des menschlichen Ohrs.
\end{enumerate}

Die resultierende Annahme aus diesen 4 Annahmen ist, dass Menschen zu einem Zeitpunkt nur einen cue pro kritischer Bandbreite dekodieren können.
